\documentclass[a4paper, 12pt]{article}

\usepackage[portuges]{babel}
\usepackage[utf8]{inputenc}
\usepackage{amsmath}
\usepackage{indentfirst}
\usepackage{blindtext}
\usepackage{graphicx}
\usepackage[hidelinks]{hyperref}
\usepackage{gensymb}
\usepackage{pgfplots}

\author{Igor Abreu da Silva}

\title{Lista Sistemas Lineares I}

\begin{document}

    \begin{titlepage}
        \begin{center}
            \huge{Universidade Federal do Rio de Janeiro}
            \vspace{95pt}

            \large{Lista II - Sistemas Lineares I}
            \vspace{160pt}
        \end{center}

        \begin{flushleft}
            \begin{tabbing}
                Alunos\qquad\qquad\= Igor Abreu da Silva\\
                DRE\> 112053874 \\
                Curso\> Engenharia Eletrônica \\
                Turma\> 2016/2 \\
                Professor\> Natanael Nunes de Moura Junior \\

            \end{tabbing}

        \end{flushleft}

        \begin{center}
            \vspace{\fill}
            Rio de Janeiro, 04 de Outubro de 2016
        \end{center}
    \end{titlepage}

    \newpage
    \tableofcontents
    \listoffigures
    \thispagestyle{empty}
    \newpage
    \pagenumbering{arabic}
    
    \section{An\'{a}lise de Sistema no Dom\'{i}nio do Tempo}
    \subsection{Quest\~{a}o 1}
    \subsubsection{Item a}
    \[\lambda^{2} + 5\lambda + 6 = 0 \rightarrow \lambda_{1} = -2; \lambda_{2} = -3 \rightarrow c_{1}e^{-2t} + c_{2}e^{-3t}\]
    \subsubsection{Item b}
    \[c_{1} + c_{2} = 2 \]
    \[-2c_{1} -3c_{2} = -1 \]
    \[c_{1} =  5\]
    \[c_{2} =  -3\]    
    \[y_{0} = 5e^{-2t} -3e^{-3t} \]
    \subsection{Quest\~{a}o 2}
    \subsubsection{Item a}   
    \[ \lambda^{2} + \lambda = 0 \rightarrow  \lambda_{1} = 0; \lambda_{2} = -1 \rightarrow c_{1}^{-2+3j} +c_{2}e^{-t}\]   
    \subsubsection{Item b}
    \[c_{1} + c_{2} = 1 \]
    \[-c_{2} = 1 \]
    \[c_{1} =  2\]
    \[c_{2} =  -1\]    
    \[y_{0} = 2 -e^{-t} \]    
    \subsection{Quest\~{a}o 3}
    \subsubsection{Item a}   
    \[\lambda^{2} + 4\lambda + 13 = 0 \rightarrow  \lambda_{1} = -2 + 3j; \lambda_{2} = -2 -3j \rightarrow c_{1}e^{(-2+3j)t} +c_{2}e^{(-2-3j)t} = ce^{-2t}cos(3t + \phi)\]   
    \subsubsection{Item b}
    \[ccos(\phi) = 1 \]
    \[-2ccos(\phi) -3csen(\phi) = 15.98 \]
    \[c =  10\]
    \[\phi =  \frac{-\pi}{3}\]    
    \[y_{0} = 10e^{-2t}cos(3t - \frac{\pi}{3}) \]      
    \subsection{Quest\~{a}o 4}
    \subsubsection{Item a}   
    \[ (\lambda + 1)(\lambda^{2} + 5\lambda + 6) = 0 \rightarrow  \lambda_{1} = -1; \lambda_{2} = -2; \lambda_{3} = -3 \rightarrow c_{1}e^{-t} +c_{2}e^{-2t} + c_{3}e^{-3t}\]   
    \subsubsection{Item b}
    \[c_{1} + c_{2} + c_{3}  = 2 \]
    \[-c_{1} -2c_{2} -3c_{3} = -1 \]
    \[c_{1} +4c_{2} +9c_{3} =  5\]
    \[c_{1} =  6\]    
    \[c_{2} =  -7\]
    \[c_{3} =  3\]
    \[y_{0} =  6e^{-t} -7e^{-2t} + 3e^{-3t}\]   
    \subsection{Quest\~{a}o 5}
    \subsection{Quest\~{a}o 6}
    \subsection{Quest\~{a}o 7} 
    \subsection{Quest\~{a}o 8}
    \[ e^{-at}u(t)\ast e^{-bt}u(t) \rightarrow \int_{0}^{t} e^{-a\tau}e^{-b(t-\tau)}\]
    \subsection{Quest\~{a}o 9}
    \subsection{Quest\~{a}o 10}
    \subsubsection{Item a} 
    \subsubsection{Item b} 
    \subsubsection{Item c} 
    \subsubsection{Item d} 
    \subsection{Quest\~{a}o 11}
    \subsubsection{Item a} 
    \subsubsection{Item b} 
    \subsubsection{Item c} 
    \subsubsection{Item d}    
    \subsection{Quest\~{a}o 12}
    \subsection{Quest\~{a}o 13}
    \subsection{Quest\~{a}o 14}
    \subsection{Quest\~{a}o 15}
    \subsubsection{Item a} 
    \subsubsection{Item b} 
    \subsubsection{Item c}     
    \subsection{Quest\~{a}o 16}
    \section{An\'{a}lise de Sistema no Dom\'{i}nio Laplace}      
    \subsection{Quest\~{a}o 1}
    \subsubsection{Item a} 
    \subsubsection{Item b} 
    \subsubsection{Item c} 
    \subsubsection{Item d} 
    \subsubsection{Item e} 
    \subsubsection{Item f} 
    \subsubsection{Item g} 
    \subsubsection{Item h}         
    \subsection{Quest\~{a}o 2}
    \subsubsection{Item a} 
    \subsubsection{Item b} 
    \subsubsection{Item c}     
    \subsection{Quest\~{a}o 3}
    \subsubsection{Item a} 
    \subsubsection{Item b} 
    \subsubsection{Item c}     
    \subsection{Quest\~{a}o 4}
    \subsubsection{Item a} 
    \subsubsection{Item b}     
    \subsection{Quest\~{a}o 5}
    \subsubsection{Item a} 
    \subsubsection{Item b}     
    \subsection{Quest\~{a}o 6}
    \subsubsection{Item a} 
    \subsubsection{Item b} 
    \subsubsection{Item c} 
	\subsubsection{Item d}     
    \subsection{Quest\~{a}o 7}
    \subsubsection{Item a} 
    \subsubsection{Item b}     
    \subsection{Quest\~{a}o 8}    
                            
\end{document}